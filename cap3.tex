\section{Aplanado de Modelos}

\begin{frame}[fragile]
\frametitle{Modelo Aplanado} 
\begin{enumerate}
\item Modelo monolítico.
\item Carencias de clases.
\item Variables de tipo básicos.
\item No posee ecuaciones \textit{connect}.

%% Todos esto es requerido para poder simularlo

\end{enumerate}
\end{frame}

\begin{frame}[fragile]
\frametitle{Simulación de Modelos Modelica} 
\begin{lstlisting}[style=base,basicstyle=\tiny]
package Circuits
    model LC_circuit
        capacitor cap(v(start = 1));
        inductor ind(L = 2);
        Pin p1,p2,p3;
    equation
        connect(ind.p,p3);
        connect(ind.p,cap.p);
        connect(cap.n,p1);
        connect(ind.n,p2);
    end LC_circuit;
    
    model LC_line
        @constant Integer N = 10;@
        @LC_circuit lc[N];@
        ground gr;
    equation
        connect(lc[N].p1,lc[N].p2)      
        for i in 1:N - 1 loop
            connect(lc[i + 1].p3,lc[i].p2);
        end for;
        for i in 1:N loop
            connect(gr.p,lc[i].p1);
        end for;
    end LC_line;
end Circuits;   
\end{lstlisting}
\end{frame}

\begin{frame}[fragile]
\frametitle{Algoritmo de Aplanado} 
\begin{lstlisting}[style=base,basicstyle=\tiny]
Flat(C):
    @2Expand(C);@
    foreach v in Variables(C):
        t = @2ResolveType(v)@;
        if isBasic(t) then 
            ChangeType(v,t);
        else if isClass(t) AND NOT isConnector(t) then
            @2ApplyModification(C,t,Modification(v));@
            Flat(t);
            @2RemoveComposition(C,t);@  
            if isConnector(t) then
                ChangeType(v,t);
            else
                Remove(t);
            end if;     
        end if;     
    end foreach;    
    foreach e in Equations(C):  
        @2ChangeVarName(e);@
    foreach e in Statements(C): 
        ChangeVarName(e);           
\end{lstlisting}
\end{frame}

\begin{frame}[fragile]
\frametitle{Expand} 
\begin{huge}
Eliminación de Herencia de clases
\end{huge} \\
La clase hijo hereda las variables y ecuaciones del padre
 
\pause 
\begin{columns} 
\column[t]{.5\textwidth}   
\begin{lstlisting}[style=base,basicstyle=\scriptsize]
model OnePort
    Pin p;
    Pin n;
    Real v;
    Real i;
equation
    v = p.v - n.v;
    i = p.i;
    i = -n.i;
end OnePort;
model Capacitor
    extends OnePort;
    parameter Real C = 1;
equation
    C * der(v) = i;
end Capacitor;
\end{lstlisting} 
\pause
\column[t]{.5\textwidth}  
\begin{lstlisting}[style=base,basicstyle=\scriptsize]
model Capacitor 
    @2Pin p;@2
    @2Pin n;@
    @2Real v;@
    @2Real i;@2
    parameter Real C = 1;
equation
    @2v = p.v - n.v;@
    @2i = p.i;@
    @2i = -n.i;@
    C * der(v) = i;
end Capacitor;
\end{lstlisting}
\end{columns}
\end{frame}



\begin{frame}[fragile]
\frametitle{ResolveType} 
Determina el tipo real de una variable y sus características: 
\begin{itemize}
    \item Prefijos de Tipos.
    \item Definición de Arreglos. 
    \item Presencia de modificaciones.
\end{itemize}

\begin{enumerate}
\item Buscar en la clase.
\item Si no esta, subo al padre. Repito.
\item Si es sinónimo, aplico ResolveType al nuevo nombre.
\item Devuelvo el tipo encontrado.
\end{enumerate}
\end{frame}

\begin{frame}[fragile]
\frametitle{ResolveType: Ejemplo} 
Definición de \textit{Voltage}
\begin{lstlisting}[style=base,literate={=}{$\leftarrow{}$}{1}{==}{$={}$}{1},escapechar=|,basicstyle=\scriptsize]
package Circuits |\only<3->{\color{blue}$\leftarrow$}|
    @4model Capacitor@
        extends OnePort;
        parameter Real C = 1;
    equation
        C * der(v) = i;
    end Capacitor;
    
    model LC_circuit |\only<2>{\color{blue}$\leftarrow$}|
        @Capacitor cap(v(start = 1));@
        inductor ind(L = 2);
        Pin p1,p2,p3;
    equation
        connect(ind.p,p3);
        connect(ind.p,cap.p);
        connect(cap.n,p1);
        connect(ind.n,p2);
    end LC_circuit;
end Circuits;
\end{lstlisting}
\end{frame}


\begin{frame}[fragile]
\frametitle{ApplyModification} 
\begin{enumerate}
\item Expandir la clase.
\item Buscar variable y agregar modificaciones.
\end{enumerate}

    Capacitor c (C=5,v(start=2))
\pause
\begin{columns} 
\column[t]{.5\textwidth}  
\begin{lstlisting}[style=base,basicstyle=\scriptsize]
model Capacitor 
    Pin p;
    Pin n;
    Real v;
    Real i;
    parameter Real C = 1;
equation
    v = p.v - n.v;
    i = p.i;
    i = -n.i;
    C * der(v) = i;
end Capacitor;
\end{lstlisting}
\column[t]{.5\textwidth}  
\begin{lstlisting}[style=base,basicstyle=\scriptsize]
model Capacitor 
    Pin p;
    Pin n;
    Real v @(start=2)@;
    Real i;
    parameter Real @C = 5@;
equation
    v = p.v - n.v;
    i = p.i;
    i = -n.i;
    C * der(v) = i;
end Capacitor;
\end{lstlisting}
\end{columns}
\end{frame}

\begin{frame}[fragile]
\frametitle{Remove Composition} 
\begin{enumerate}
\item Elimina instancia de clase.
\item Añade las variables y ecuaciones internas.
\item Renombrar las variables.
\item Si la instancia está vectorizada: 
    \begin{enumerate}
    \item Agrega las variables vectorizadas.
    \item Encapsulamos las ecuaciones dentro de una ecuación \textit{for}.
    \end{enumerate} 
\end{enumerate}

\end{frame}

\begin{frame}[fragile]
\frametitle{Remove Composition: Variables} 
\begin{enumerate}
\item Agregamos un prefijo al nombre de las variables: "nombreInstancia\_".
\item Mantenemos prefijos de variable. 
\item Mantenemos las definiciones de arreglos y agregamos nuevas si la instancia lo está.   
\end{enumerate}

\pause
\begin{columns} 
\column[t]{.5\textwidth}  
\begin{lstlisting}[style=base,basicstyle=\scriptsize]
model Capacitor 
    Real p_v;
    flow Real p_i;
    Real n_v;
    flow Real n_i; 
    Real v;
    Real i;
    parameter Real C = 1;
equation 
    ...
end Capacitor;
model LC_circuit
    @capacitor cap(v(start = 1));@
    inductor ind(L = 2);
    Pin p1,p2,p3;
equation
    ...
end LC_circuit;
\end{lstlisting}
\column[t]{.5\textwidth}  
\begin{lstlisting}[style=base,basicstyle=\scriptsize]
model LC_circuit
    @2Real cap_p_v;@
    @2flow Real cap_p_i;@
    @2Real cap_n_v;@
    @2flow Real cap_n_i;@ 
    @2Real cap_v (start = 1);@
    @2Real cap_i;@
    @2parameter Real cap_C = 1;@
    
    inductor ind(L = 2);
    Pin p1,p2,p3;
equation
    ...
end LC_circuit;
\end{lstlisting}
\end{columns}


\end{frame}


