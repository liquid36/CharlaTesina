\begin{frame}[fragile]
\frametitle{Algoritmo de aplanado} 
\begin{center}
\huge Resolución de conexiones
\end{center}
\end{frame}

\begin{frame}[fragile]
\frametitle{Resolución de conexiones} 
\begin{block}{En un principio}
\begin{enumerate}
\item Expandir las ecuaciones \textit{for}.
\item Generar de un grafo a partir de las ecuaciones \textit{connects}.
\item Determinar las componentes conexas del grafo generado.
\item Generación de ecuaciones a partir de las componentes conexas.
\end{enumerate}
\end{block}

\pause
\begin{block}{Proponemos}
\begin{enumerate}
\item Grafo bipartito cuyos nodos y aristas están etiquetadas según las ecuaciones. \\Grafo Vectorizado.
\item Búsqueda en profundida modificada para determinar las componentes conexas.    
\item Generación de ecuaciones a partir de las componentes conexas.
\end{enumerate}
\end{block}

\end{frame}

\begin{frame}{fragile}
\frametitle{Grafo Vectorizado}
\begin{enumerate}
\setlength\itemsep{1em}
\item Agregamos un nodo por cada variable de la ecuación.
\item Agregamos un nodo que representa a la ecuación \textit{connect}.
\item Si la ecuación estaba dentro de un \textit{for}, etiquetamos el nodo \textit{connect} con el rango de iteraci\'on. 
\item Agregamos dos aristas, entre cada variable y el nodo que representa al \textit{connect}.
\item Por cada arista, si la variable asociada tiene un índice de acceso, agregamos esa referencia a la arista. 
\end{enumerate}
\end{frame}

\begin{frame}[fragile]
\frametitle{Grafo Bipartito: Ejemplo}
\begin{columns} 
\column[t]{7cm}  
\begin{lstlisting}[style=base]
connect(lc_p1[N],lc_p2[N])     
 
for i in 1:N - 1 loop
    connect(lc_p3[i + 1],lc_p2[i]);
end for;

for i in 1:N loop
    connect(gr_p[i],lc_p1[i]);
end for;
\end{lstlisting}

\column[t]{7cm}  

\begin{tikzpicture}[>=stealth',shorten >=1pt,auto, node distance = 1.5cm,
                    thick,main node/.style={circle,draw,font=\sffamily\bfseries}]
 
  \node[main node] (4)  {gr\_p};
  \node[main node] (1) [below of=4] {lc\_p1};
  \node[main node] (2) [below of=1] {lc\_p2};
  \node[main node] (3) [below of=2] {lc\_p3};
  
  
  \node[main node] (7) [right of = 4, xshift=3cm]  {1:N};
  \node[main node] (5) [below of=7] {\{\}};
  \node[main node] (6) [below of=5]  {1:N-1};
    
    
  \path[every node/.style={font=\sffamily\small}]   
    (1) edge node [left,near start,above] {N} (5)
        edge node[left,very near start,above] {i} (7)
    (2) edge node [left,pos = 0.05,above] {N} (5)
        edge node[left,very near start,below] {i} (6)  
    (3) edge node [left,near start,above] {i + 1} (6)    
    (4) edge node [left,near start,above] {i} (7);
\end{tikzpicture}
\end{columns}
\end{frame}

\begin{frame}{fragile}
\frametitle{Determinación de componentes conexas}
\begin{itemize}
\setlength\itemsep{1em}
\item Búsqueda en profundida sobre el grafo. 
\item Cada visita a un nodo depende de la metadata del grafo.
\item Múltiples visitas a un mismo nodo. 
\item Mantenemos referencia del intervalo de acceso al nodo. 
\item Visitamos un nodo si hay intersección entre intervalos.
\item Podamos parte de la arista al visitar un nodo.
\item Terminamos cuando no hay más intersección con otros nodos.
\end{itemize}
\end{frame}


\begin{frame}[fragile]
\frametitle{Determinación de componentes conexas}
\begin{columns} 
\column[t]{7cm}  
  \newcommand*\nstart{} 
  \newcommand*\none{}
  \newcommand*\ntwo{}
  \newcommand*\nfive{}
  \newcommand*\nsixe{}
  \newcommand*\nocho{}
  \only<1-2>{\renewcommand*\nstart{green}}
  \only<2->{\renewcommand*\none{blue}}
  \only<3-5>{\renewcommand*\ntwo{green}}
  \only<5->{\renewcommand*\nfive{blue}}
  \only<6->{\renewcommand*\nsixe{green}}
  \only<8>{\renewcommand*\nocho{blue}}

\begin{tikzpicture}[>=stealth',shorten >=1pt,auto, node distance = 1.5cm,
                    thick,main node/.style={circle,draw,font=\sffamily\bfseries}]  
                    
  \node[main node, \nstart] (4)  {gr\_p};
  \node[main node,\ntwo] (1) [below of=4]{lc\_p1};
  \node[main node,\nsixe] (2) [below of=1] {lc\_p2};
  \node[main node] (3) [below of=2] {lc\_p3};
  
  
  \node[main node] (7) [right of = 4,xshift=3cm]  {1:N};
  \node[main node] (5) [below of=7 ] {\{\}};
  \node[main node] (6) [below of=5]  {1:N-1};
    
    
  \path<1-6>[every node/.style={font=\sffamily\small}]  
    (1) edge[\nfive] node [left,near start,above] {N} (5)
    (2) edge[\nfive] node [left,pos = 0.05,above] {N} (5);
    
   \path[every node/.style={font=\sffamily\small}]    
     (2)   edge[\nocho] node[left,very near start,below] {i} (6)  
    (3) edge node [left,near start,above] {i + 1} (6);
    
   \path<1-3>[every node/.style={font=\sffamily\small}]     
        (1) edge[\none] node[left,very near start,above] {i} (7)    
        (4) edge[\none] node [left,near start,above] {i} (7);
        
\end{tikzpicture}

\column[t]{8cm}  
\begin{enumerate}
\item Arrancamos en \textit{gr\_p}.
\item<2-> Existe único camino y tiene intervalo 1:N.
\item<3-> Visitamos \textit{lc\_p1} con intervalo 1:N.
\item<4-> Podamos el grafo.
\item<5-> Existe un camino con índice N.
\item<6-> Visitamos \textit{lc\_p2} con índice N.
\item<7-> Podamos el grafo.
\item<8-> $N \cap 1:N-1 = \{\}$. \\
          No hay más nodos para buscar.

\end{enumerate}
\end{columns}
\end{frame}

\begin{frame}[fragile]
\frametitle{Determinación de componentes conexas}
\begin{columns} 
\column[t]{7cm}  

  \newcommand*\nstart{} 
  \newcommand*\none{} 
  \newcommand*\ntwo{} 
  \only<1>{\renewcommand*\nstart{green}}
  \only<2-3>{\renewcommand*\none{green}} 
  \only<4->{\renewcommand*\ntwo{green}} 

\begin{tikzpicture}[>=stealth',shorten >=1pt,auto, node distance = 1.5cm,
                    thick,main node/.style={circle,draw,font=\sffamily\bfseries}]  
  
  \node[main node,\ntwo] (4)  {gr\_p};
  \node[main node,\none] (1) [below of=4]{lc\_p1};
  \node[main node,\nstart] (2) [below of=1] {lc\_p2};
  \node[main node] (3) [below of=2] {lc\_p3};
  
  
  \node[main node] (7) [right of = 4,xshift=3cm]  {1:N};
  \node[main node] (5) [below of=7] {\{\}};
  \node[main node] (6) [below of=5]  {1:N-1};
     
    
   \path[every node/.style={font=\sffamily\small}]    
    (2) edge node[left,very near start,below] {i} (6)  
    (3) edge node [left,near start,above] {i + 1} (6); 
        
\end{tikzpicture}

\column[t]{8cm}  
\begin{enumerate}
\setlength\itemsep{2em}
\item En \textit{lc\_p2} la solución es:    
    \begin{itemize}
    \item $\langle \ \textit{lc\_p2[N]} \ \rangle$. 
    \end{itemize}

\item<2-> En \textit{lc\_p1} hubo una partición de intervalos:
    \begin{itemize}
    \item $1:N-1$ \only<3->{$\rightarrow \  \langle \textit{lc\_p1[i]} \  \rangle$ en $1:N-1$ }
    \item $N$     \only<3->{$\rightarrow \  \langle \textit{lc\_p1[N]},\ \textit{lc\_p2[N]} \  \rangle$ }
    \end{itemize}
    
\item<4-> En \textit{gr\_p} las soluciones finales son:
    \begin{itemize}
    \item $\langle \  \textit{gr\_p[i]},\ \textit{lc\_p1[i]} \  \rangle$ en $1:N-1$
    \item $\langle \  \textit{gr\_p[N]},\ \textit{lc\_p1[N]},\ \textit{lc\_p2[N]} \  \rangle$  
    \end{itemize}
    
\end{enumerate}
\end{columns}
\end{frame}



\begin{frame}{fragile}
\frametitle{Generación de ecuaciones}
\begin{itemize}
\setlength\itemsep{1em}
\item Cada componente conexa determina un conjunto de ecuaciones.
\item Determinar el tipo de conector usado.
\item Las variables de potencial deben quedar igualadas entre sí.
\item Las variables de flujo deben sumar cero. 
\end{itemize}
\end{frame}

\begin{frame}[fragile]
\frametitle{Generación de ecuaciones} 
\begin{itemize}
\item $\langle \ \textit{gr\_p[i]},\ \textit{lc\_p1[i]} \ \rangle$ en $1:N-1$. \\
\vspace{1em}
\begin{lstlisting}[style=base]
    @2for i in 1:N - 1 loop@
        gr_p_v[i] = lc_p1_v[i];
        gr_p_i[i] + lc_p1_i[i] = 0;
    @2end for;@
\end{lstlisting}

\pause
\item $\langle \ \textit{gr\_p[N]},\ \textit{lc\_p1[N]},\ \textit{lc\_p2[N]} \ \rangle$ . \\
\vspace{1em}
\begin{lstlisting}[style=base]
    gr_p_v[N] = lc_p1_v[N];  
    gr_p_v[N] = lc_p2_v[N];
    gr_p_i[N] + lc_p2_i[N] + lc_p2_i[N] = 0;
\end{lstlisting}
\end{itemize}
\end{frame}
