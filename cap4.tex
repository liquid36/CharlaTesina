\begin{frame}[fragile]
\frametitle{Algoritmo de aplanado} 
\begin{center}
\huge Resolución de conexiones
\end{center}
\end{frame}



\begin{frame}[fragile]
\frametitle{Resolución de conexiones} 
\begin{block}{Resolución de conexiones}
\begin{enumerate}
\item Generación de un grafo bipartito a partir de las ecuaciones \textit{connects}.
\item Determinación de componente conexas del grafo generado.
\item Generación de ecuaciones a partir de las componentes conexas.
\end{enumerate}
\end{block}
\end{frame}

\begin{frame}{fragile}
\frametitle{Grafo bipartito}
\begin{enumerate}
\setlength\itemsep{1em}
\item Agregamos un nodo por cada variable de la ecuación.
\item Agregamos un nodo que representa a la ecuación \textit{connect}.
\item Si la ecuación estaba dentro de un \textit{for}, etiquetamos el nodo \textit{connect} con el rango de iteraci\'on. 
\item Agregamos dos aristas, entre cada variable y el nodo que representa al \textit{connect}.
\item Por cada arista, si la variable asociada tiene un índice de acceso, agregamos esa referencia a la arista. 
\end{enumerate}
\end{frame}

\begin{frame}[fragile]
\frametitle{Grafo Bipartito: Ejemplo}
\begin{columns} 
\column[t]{7cm}  
\begin{lstlisting}[style=base]
connect(lc_p1[N],lc_p2[N])     
 
for i in 1:N - 1 loop
    connect(lc_p3[i + 1],lc_p2[i]);
end for;

for i in 1:N loop
    connect(gr_p,lc_p1[i]);
end for;
\end{lstlisting}

\column[t]{7cm}  

\begin{tikzpicture}[>=stealth',shorten >=1pt,auto, node distance = 1.5cm,
                    thick,main node/.style={circle,draw,font=\sffamily\bfseries}]
 
  \node[main node] (1) {lc\_p1};
  \node[main node] (2) [below of=1] {lc\_p2};
  \node[main node] (3) [below of=2] {lc\_p3};
  \node[main node] (4) [below of=3] {gr\_p};
  
  \node[main node] (5) [right of = 1,xshift=3cm ] {\{\}};
  \node[main node] (6) [below of=5]  {1:N-1};
  \node[main node] (7) [below of=6]  {1:N};  
	
  \path[every node/.style={font=\sffamily\small}] 	
	(1) edge node [left,near start,above] {N} (5)
        edge node[left,very near start,above] {i} (7)
    (2) edge node [left,pos = 0.05,above] {N} (5)
        edge node[left,very near start,below] {i} (6)  
    (3) edge node [left,near start,above] {i + 1} (6)    
	(4) edge node [left,near start,above] {1} (7);
\end{tikzpicture}


\end{columns}
\end{frame}

\begin{frame}{fragile}
\frametitle{Determinación de componentes conexas}
\begin{itemize}
\setlength\itemsep{1em}
\item Busqueda en profundida sobre el grafo. 
\item Cada visita a un nodo depende de las metadata del grafo.
\item Mantenemos referencia del intervalo de acceso al nodo. 
\item Visitamos un nodo si hay intersección entre intervalos.
\item Podamos parte de la arista al visitar un nodo.
\item Terminamos cuando no hay más intersección con otros nodos.
\end{itemize}
\end{frame}
