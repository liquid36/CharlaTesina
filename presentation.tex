\documentclass[aspectratio=169,10pt]{beamer}
\mode<presentation>
{
  %%\usetheme{Copenhagen}
  \usetheme{Warsaw}
  \setbeamercovered{transparent}
}

\setcounter{tocdepth}{1}
\AtBeginSection[]
{
  \begin{frame}<beamer>
    \frametitle{Contenido de la charla}
    \tableofcontents[currentsection,currentsubsection]
  \end{frame}
}
\usepackage[latin1]{inputenc}
\usepackage{times}
\usepackage[T1]{fontenc}
\usepackage[scanall]{psfrag}
\usepackage{bibunits}
\usepackage[spanish]{babel} 
\usepackage{listings}
\usepackage{xcolor}
\usepackage{pgf,pgffor} 

\lstdefinestyle{base}{  breaklines=true,
                    basicstyle=\footnotesize,
                    numbers=left,
                    numberstyle=\tiny, numbersep=5pt,
                    breaklines=true,
                    tabsize=2, captionpos=b , stepnumber=1 , aboveskip=20pt,
                    moredelim=**[is][\color{red}]{@}{@}
        }

\title{Aplanado eficiente de grandes modelos Modelica}
\author[M.Botta] {Mariano Botta } 
\institute[UNR] % (optional, but mostly needed)
{ FCEIA, UNR }
\date {Agosto 2015}

\subject{Talks}
\begin{document}

\begin{frame}
  \titlepage
\end{frame}

 
\section{Motivaciones}  

\begin{frame}{Motivaciones}
    \begin{itemize}
     \item Modelado, Simulaci\'on y Control en Tiempo Real con Aplicaciones en Electr\'onica de Potencia.
     \item Simulaci\'on en paralelo utilizando los m\'etodos de cuantificaci\'on de estado.  
     \item Modelos grandes.
     \item Aprovechar las ventajas de Modelica.
    \end{itemize}
\end{frame}

\section{Introducci\'on a Modelica}

\begin{frame}{Modelica} 
    \begin{itemize}
        \item Orientado a Objetos.
        \item Modelado de sistemas complejos, con componentes mec\'anicos, el\'ectricos, electr\'onicos, hidr\'aulicos, t\'ermicos, etc.     
        \item Desarrollado por la asociaci\'on sin fines de lucro ``Modelica Asociation''.
        \item Entornos de desarrollo: OpenModelica, MathModelica, Dymola, etc.
        
        \item El modelo esta en texto plano
    \end{itemize}
\end{frame}
 

\begin{frame}[fragile]
\frametitle{Clases} 
\begin{columns}  
\column{.5\textwidth}  
\begin{itemize}
    \item Define un objeto.
    \item Son instanciadas mediante la definici\'on de variables.
    \item Tienen tres secci\'ones:
        \begin{enumerate}
            \item Definiciones.
            \item Ecuaciones.
            \item Sentencias.
        \end{enumerate}
\end{itemize}

\column{.5\textwidth}
\begin{lstlisting}[style=base]
    class X
        // Definiciones de variables y clases   
    equation
        // Ecuaciones
    statements
        // Sentencias
    end X;   
\end{lstlisting}
\end{columns}
\end{frame}

\begin{frame}[fragile]
\frametitle{Prefijos de Clases} 
Prefijos de clase: \textit{model}, \textit{record}, \textit{block}, \textit{connector}, \textit{function}, \textit{package}
\begin{itemize}
    \item Mejoran la lectura del c\'odigo:
    \item Agregan restricciones a la clase
\end{itemize}   

\begin{columns}  
\column{.5\textwidth}  
\begin{lstlisting}[style=base]
    class Circuits
        cclass Pin
            Real v;
            flow Real i;
        end Pin;
        class Componente
            Pin n,p;
        equation 
            n.v = p.v;
        end Componente; 
    end Circuits;   
\end{lstlisting}
\par

\column{.5\textwidth}
\begin{lstlisting}[style=base]
    @package@ Circuits
        @connector@ Pin
            Real v;
            flow Real i;
        end Pin;
        @model@ Componente
            Pin n,p;
        equation 
            n.v = p.v;
        end Componente; 
    end Circuits;   
\end{lstlisting}
\par
\end{columns}
\end{frame}

\begin{frame}[fragile]
\frametitle{Herencias de Clases} 
\begin{columns}  
\column[t]{.5\textwidth}  

\begin{itemize}
    \item Agrega significado semantico al modelo.
    \item Facilita la reutilizaci\'on de c\'odigo.
    \item Se utiliza la palabra reservada: \textit{extends}.
\end{itemize} 

\column[t]{.5\textwidth}  
\begin{lstlisting}[style=base]
model OnePort
    Pin p;
    Pin n;
    Real v;
    Real i;
equation
    v = p.v - n.v;
    i = p.i;
    i = -n.i;
end OnePort;
model Capacitor
    @extends@ OnePort;
    parameter Real C = 1;
equation
    C * der(v) = i;
end Capacitor;
\end{lstlisting}
\end{columns}
\end{frame}

\begin{frame}[fragile]
\frametitle{Declaraciones de tipo} 
\begin{columns}  
\column[t]{8cm}
    \begin{itemize}
        \item Tipos b\'asicos: \textit{Real}, \textit{Integer}, \textit{Boolean} y \textit{String}
        \item Las clases definen un nuevo tipo.
        \item Sin\'onimos de tipos: \\ 
              type Nombre = [Prefijos] Tipo-Existente [Array] [Modificaciones]
    \end{itemize} 

Prefijos de Tipo: \textit{flow}, \textit{constant}, \textit{parameter}, \textit{discrete}, \textit{input} y \textit{output}.

\column[t]{5cm}
\begin{lstlisting}[style=base]
package Circuits
    type Current = flow Real;
    type Voltage = Real;
    connector Pin
        Voltage v;
        Current i;
    end Pin;
    type TenPin = Pin[10];
end Circuits;
\end{lstlisting}
\par
\end{columns}
\end{frame}

\begin{frame}[fragile]
\frametitle{Definiciones de variables} 
\begin{enumerate}
\item \textbf{Prefijos de tipos}: \textit{flow}, \textit{constant}, \textit{parameter}, \textit{discrete}, \textit{input} y \textit{output}.
\item \textbf{Tipo}: Nombre del tipo de la variable. Puede ser un tipo b\'asico, una clase o un sin\'onimo de tipo.  Ejemplo: \textit{Real}, \textit{String}, \textit{Pin}, \textit{TenPin}.
\item \textbf{Nombre de la variable}. 
\item \textbf{Dimensi\'on}: Modelica permite la definici\'on de arreglos.
\item \textbf{Modificaciones}.
\end{enumerate}
\begin{lstlisting}[style=base]
    type TenPin = Pin[10];  
    
    TenPin pines;
    Pin pines2 [10];
\end{lstlisting}
\par

\end{frame}

\begin{frame}[fragile]
\frametitle{Modificaciones} 

\begin{columns}  
\column[T]{7cm}
Aparecen en: 
\begin{enumerate}
    \item Declaraciones de variables.
    \item Sin\'onimo de tipo.
    \item Definiciones de herencia.
\end{enumerate} 

Se permite:
\begin{enumerate}
    \item Cambiar el valor inicial de una variable.
    \item Anidar modificaciones.
    \item Alterar la definici\'on de una variable.
    \item Cambiar la definici\'on de un tipo.
\end{enumerate}
\column[T]{7cm}
\begin{lstlisting}[style=base]
package Circuits
    model CircuitX
        Capacitor cap;
        Resistor res;
        ...
    equation 
    
    ...
        
    end CircuitX;
    model MainCircuit
        Capacitor x(C = 2);
        CircuitOne co1 (cap(C = 10));
        CircuitOne co2 (cap(C = 15));
    end MainCircuit;
end Circuits;
\end{lstlisting}

 
\end{columns}
\end{frame}

\begin{frame}[fragile]
\frametitle{Ecuaciones de igualdad} 

Las ecuaciones no representan una asignaci\'on, sino igualdades. Pueden tener expresiones complejas de ambos lados de la igualdad y expresan una relaci\'on entre las variables. Se definen dentro de la secci\'on \textit{equation}.

\begin{columns}  
\column[t]{8cm}
Igualdades
\begin{lstlisting}[style=base]
    p.v - n.v = v;
    i = p.i;
    i = -n.i;
\end{lstlisting} 
\column[t]{5cm}
Ecuac\'on For
\begin{lstlisting}[style=base]
    for i in 1:N loop
        v[i] = p[i].v - n[i].v;
        i[i] = p[i].i;
        i[i] = -n[i].i;
        C[i] * der(v[i]) = i;
    end for;
\end{lstlisting} 
\end{columns}


\end{frame}

\begin{frame}[fragile]
\frametitle{Ecuaciones Connect} 
Los conectores:
\begin{itemize}
\item Son clases con ciertas restricciones.
\item Se definen con el prefijo \textit{connector}.
\item No tienen ecuaciones.
\item Tienen variables de dos tipos:
    \begin{enumerate}
        \item Variables de potencial. Ejemplo: presi\'on, voltaje, etc. 
        \item Variables de flujo: definidas con el prefijo flow. Ejemplo: corriente, caudal, etc.
    \end{enumerate}
\item Ejemplo: Clase \textit{Pin}.
\end{itemize}

\end{frame}

\begin{frame}[fragile]
\frametitle{Ecuaciones Connect} 
Las ecuaciones \textit{connect}: 
\begin{itemize}
\item Conectan dos clases del mismo tipo.
\item Genera relaciones entre las variables internas de los conectores:
\begin{itemize}
\item Las variables de potencial dentro de una misma conexi\'on deben ser iguales entre s\'i.
\item Las variables de flujo siguen las reglas de Kirchhoff: la suma de los flujos es igual a cero. Para mantener esta regla hay que considerar como flujo positivo aquel que tenga direcci\'on hacia dentro del componente. En caso contrario, ser\'a considerado negativo. 
\end{itemize}
\end{itemize}
\end{frame}

\begin{frame}[fragile]
\frametitle{Ecuaciones Connect} 
\begin{columns}  
\column[T]{8cm}
 \begin{lstlisting}[basicstyle=\ttfamily\scriptsize,style=base]
    package Circuits
        ...
        model ground
            Pin p;
        equation
            p.v = 0;
        end ground;
        model inductor
            extends OnePort;
            parameter Real L = 1;
        equation
            L * der(i) = v;
        end inductor;
        model LC_circuit
            Capacitor cap(v(start = 1));
            inductor ind(L = 2);
            ground gr;
        equation
            connect(ind.p,cap.p);
            connect(ind.n,cap.n);
            connect(cap.n,gr.p);
        end LC_circuit;
    end Circuits
\end{lstlisting}
\column[T]{6cm}
 \begin{lstlisting}[style=base]
    // Variables de Potencial
    ind.p.v = cap.p.v;
    ind.n.v  = cap.n.v;
    cap.n.v = gr.p.v;
    
    // Variables de flujo
    ind.p.i + cap.p.i = 0;
    ind.n.i + cap.n.i + gr.p.i = 0;
\end{lstlisting}
\end{columns}
\end{frame}

\section{Simulaci\'on y problemas}

\begin{frame}[fragile]
\frametitle{Simulaci\'on de Modelos Modelica} 
    \begin{figure}
      \centering
      \includegraphics[scale=0.5]{Compilacion} 
      \label{fig:proceso}
    \end{figure}
\end{frame}


%% [FALTA] Esta no estoy seguro
\begin{frame}[fragile]
\frametitle{Simulaci\'on de Modelos Modelica} 
 Las ecuaciones de un modelo deben ser representadas como Ecuaciones Diferenciales Ordinarias (ODE): 
 \begin{enumerate}
 \item \textbf{Aplanado del modelo}.
 \item Reducci\'on de \'indices.
 \item Ordenamiento y optimizaci\'on de ecuaciones.
 \end{enumerate}
 
\end{frame}

\begin{frame}[fragile]
\frametitle{Tama\~no de un modelo} 
\begin{itemize}
    \item Tama\~no f\'isico: Cantidad de clases, variables, ecuaciones, etc usadas.
    \item Tama\~no l\'ogio: consideramos la dimensionalidad del modelo.
\end{itemize}
 
\uncover<2->{ 
\begin{center} 
    \LARGE Queremos trabajar con grandes dimensionalidades 
\end{center}
}
\end{frame}



\begin{frame}[fragile]
\frametitle{Simulaci\'on de Modelos Modelica} 
\begin{lstlisting}[style=base]
package Circuits
    model LC_circuit
        capacitor cap(v(start = 1));
        inductor ind(L = 2);
        Pin p1,p2,p3;
    equation
        connect(ind.p,p3);
        connect(ind.p,cap.p);
        connect(cap.n,p1);
        connect(ind.n,p2);
    end LC_circuit;
    
    model LC_line
        @constant Integer N = 10;@
        @LC_circuit lc[N];@
        ground gr;
    equation
        connect(lc[N].p1,lc[N].p2)      
        for i in 1:N - 1 loop
            connect(lc[i + 1].p3,lc[i].p2);
        end for;
        for i in 1:N loop
            connect(gr.p,lc[i].p1);
        end for;
    end LC_line;
end Circuits;   
\end{lstlisting}
\end{frame}

\end{document}
    



\begin{comment}

\begin{frame}[fragile]
\frametitle{Declaraciones de variables} 
\begin{columns}  
\column[t]{8cm}
 
\column[t]{5cm}
 
\end{columns}
\end{frame}

\end{comment}
