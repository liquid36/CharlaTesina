\documentclass[aspectratio=169,10pt]{beamer}
\mode<presentation>
{
  %%\usetheme{Copenhagen}
  \usetheme{Warsaw}
  \setbeamercovered{transparent}
}

\setcounter{tocdepth}{1}
\AtBeginSection[]
{
  \begin{frame}<beamer>
    \frametitle{Contenido de la charla}
    \tableofcontents[currentsection,currentsubsection]
  \end{frame}
}
\usepackage[latin1]{inputenc}
\usepackage{times}
\usepackage[T1]{fontenc}
\usepackage[scanall]{psfrag}
\usepackage{bibunits}
\usepackage[spanish]{babel} 
\usepackage{listings}
\usepackage{xcolor}
\usepackage{pgf,pgffor} 

\lstdefinestyle{base}{  breaklines=true,
                    basicstyle=\footnotesize,
                    numbers=left,
                    numberstyle=\tiny, numbersep=5pt,
                    breaklines=true,
                    tabsize=2, captionpos=b , stepnumber=1 , aboveskip=20pt,
                    moredelim=**[is][\color{red}]{@}{@}
        }

\title{Aplanado eficiente de grandes modelos Modelica}
\author[M.Botta] {Mariano Botta } 
\institute[UNR] % (optional, but mostly needed)
{ FCEIA, UNR }
\date {Agosto 2015}

\subject{Talks}
\begin{document}

\begin{frame}
  \titlepage
\end{frame}

 
\section{Motivaciones}  

\begin{frame}{Motivaciones}
    \begin{itemize}
     \item Modelado, Simulaci\'on y Control en Tiempo Real con Aplicaciones en Electr\'onica de Potencia.
     \item Simulaci\'on en paralelo utilizando los m\'etodos de cuantificaci\'on de estado.  
     \item Modelos grandes.
     \item Aprovechar las ventajas de Modelica.
    \end{itemize}
\end{frame}

\section{Introducci\'on a Modelica}



\begin{frame}{Modelica} 
    \begin{itemize}
        \item Orientado a Objetos.
        \item Modelado de sistemas complejos, con componentes mec\'anicos, el\'ectricos, electr\'onicos, hidr\'aulicos, t\'ermicos, etc.     
        \item Desarrollado por la asociaci\'on sin fines de lucro ``Modelica Asociation''.
        \item Entornos de desarrollo: OpenModelica, MathModelica, Dymola, etc.
        
        \item El modelo esta en texto plano
    \end{itemize}
\end{frame}
 

\begin{frame}[fragile]
\frametitle{Clases} 
\begin{columns}  
\column{.5\textwidth}  
\begin{itemize}
    \item Define un objeto.
    \item Son instanciadas mediante la definici\'on de variables.
    \item Tienen tres secci\'ones:
        \begin{enumerate}
            \item \only<2->{Definiciones}
            \item Ecuaciones
            \item Sentencias
        \end{enumerate}
\end{itemize}

\column{.5\textwidth}
\begin{lstlisting}[style=base]
    class X
        // Definiciones de variables y clases   
    equation
        // Ecuaciones
    statements
        // Sentencias
    end X;   
\end{lstlisting}
\end{columns}
\end{frame}

\begin{frame}[fragile]
\frametitle{Prefijos de Clases} 
Prefijos de clase: \textit{model}, \textit{record}, \textit{block}, \textit{connector}, \textit{function}, \textit{package}
\begin{itemize}
    \item Mejoran la lectura del c\'odigo:
    \item Agregan restricciones a la clase
\end{itemize}   

\begin{columns}  
\column{.5\textwidth}  
\begin{lstlisting}[style=base]
    class Circuits
        cclass Pin
            Real v;
            flow Real i;
        end Pin;
        class Componente
            Pin n,p;
        equation 
            n.v = p.v;
        end Componente; 
    end Circuits;   
\end{lstlisting}
\par

\column{.5\textwidth}
\begin{lstlisting}[style=base]
    @package@ Circuits
        @connector@ Pin
            Real v;
            flow Real i;
        end Pin;
        @model@ Componente
            Pin n,p;
        equation 
            n.v = p.v;
        end Componente; 
    end Circuits;   
\end{lstlisting}
\par
\end{columns}
\end{frame}

\end{document}